\documentclass[10pt,a4paper]{article}
\usepackage[utf8]{inputenc}
\usepackage[spanish]{babel}
\usepackage{amsmath}
\usepackage{amsfonts}
\usepackage{amssymb}
\author{ }
\title{Anàlisi Complexa - Laboratori 1}
\begin{document}
\maketitle

\begin{enumerate}
\item \textbf{Demostreu que per tot $z\in\mathbb{C}$
$$|Re(z)|+|Im(z)|\leq\sqrt{2}|z|$$}
\underline{Resolució:}

Sigui $z$ un nombre complex de la forma $z=x+iy=|z|(\cos(\alpha)+i\sin(\alpha))$. Amb aquestes notacions, $|Re(z)|=|x|$ i $|Im(z)|=|y|$.

Ara bé, tenim que:
$$|x| = |z||\cos(\alpha)|$$
$$|y| = |z||\sin(\alpha)|$$

A més a més,

$$|x|+|y| = |z|(|\cos(\alpha)| + |\sin(\alpha)|)$$

Per arribar a la desigualtat que volem, busquem els màxims de l'expressió $|\cos(\alpha)| + |\sin(\alpha)|$. Per fer-ho, notem que podem treure els valors absoluts i suposar que $\alpha$ és del primer quadrant per la periodicitat de les funcions $\sin$ i $\cos$.

Definim llavors $f(\alpha)=\cos(\alpha) + \sin(\alpha)$. Derivem i igualem a 0:
$$f'(\alpha) = \cos(\alpha)-\sin(\alpha)=0 \Leftrightarrow \cos(\alpha)=\sin(\alpha)$$ que ens dóna la solució $\alpha=\frac{\pi}{4}$. Podem comprovar que és el màxim absolut.

Tenim llavors que $$|x|+|y| = \left|z\right|\left(\left|\cos\left(\alpha\right)\right| + \left|\sin\left(\alpha\right)\right|\right) \leq \left|z\right|\left(\cos\left(\frac{\pi}{4}\right) + \sin\left(\frac{\pi}{4}\right)\right) =$$ 
$$= |z|\left(\frac{\sqrt{2}}{2}+\frac{\sqrt{2}}{2}\right) = |z|\sqrt{2}$$

Com volíem veure.

\item \textbf{Sigui $\Omega$ un domini i $f$ una funció holomorfa en $\Omega$. Proveu que les següents condicions són equivalents.}

\begin{enumerate}
\item \textbf{$Re(f)$ és constant en $\Omega$}
\item \textbf{$Im(f)$ és constant en $\Omega$}
\item \textbf{La funció conjugada de $f$, $\bar{f}$, és holomorfa en $\Omega$}
\item \textbf{$f$ és constant en $\Omega$}
\item \textbf{$|f|$ és constant en $\Omega$}
\end{enumerate}

\underline{Resolució:}

Per veure les equivalències, es veu trivialment que l'enunciat d) implica tota la resta de enunciats. Ara veïem les implicacions contràries:

\begin{enumerate}
\item $(a)\Rightarrow(d)$:

Tenim que $f(x+yi)=k_{1}+v(x,y)i$, $k_{1}\in\mathbb{C}$
	
Llavors $$Df(x,y)=
\left(\begin{matrix}
  0 & 0 \\
  v_{x}(x,y) & v_{y}(x,y)
\end{matrix}\right) = 
\left(\begin{matrix}
  \alpha & -\beta \\
  \beta & \alpha
\end{matrix}\right)$$

D'aquí veiem que $v_{x}(x,y)=v_{y}(x,y)=0 \Leftrightarrow v(x,y)=k_{2}\in\mathbb{C}$

Com volíem veure.

\item $(b)\Rightarrow(d)$:

Es pot veure anàlogament, però en aquest cas és la segona fila de la matriu jacobiana que és de 0s.

\item $(c)\Rightarrow(d)$:

Si $Df(x,y)=
\left(\begin{matrix}
  \alpha & -\beta \\
  \beta & \alpha
\end{matrix}\right)$
Tenim que:
$D\bar{f}(x,y)=
\left(\begin{matrix}
  \alpha & -\beta \\
  -\beta & -\alpha
\end{matrix}\right)$

Perquè la matriu compleixi les equacions de Cauchy Riemann, tenim que $\alpha=-\alpha$ i $-\beta=\beta$ que només pot passar si $\alpha=\beta=0$ $\Rightarrow$ $f$ és constant.

\item $(e)\Rightarrow(d)$:

Si $|f|$ és constant també l'és $|f|^{2}(x,y) = u(x,y)^2 + v(x,y)^2 = K \in \mathbb{C}$

Llavors 

$$\frac{\partial (u^{2} + v^{2}) }{\partial x} = 0 = \dfrac{\partial (u^{2} + v^{2}) }{\partial y}$$

Això implica que:

$$u \frac{\partial u}{\partial x} + v \frac{\partial v}{\partial x} = 0 = u \frac{\partial u}{\partial y} + v \frac{\partial v}{\partial y}$$

Si apliquem les equacions de Cauchy-Riemann arribem a:

$$u \frac{\partial v}{\partial y} + v \frac{\partial v}{\partial x} = 0 = -u \frac{\partial v}{\partial x} + v \frac{\partial v}{\partial y}$$

$$(u^{2} + v^{2}) \frac{\partial v}{\partial y} = 0$$

Si $u^{2}+v^{2}=0$, llavors $f=0$ i ja tenim que $f$ és constant. 

Si $u^{2}+v^{2}\neq 0$, llavors $\dfrac{\partial v}{\partial y}=0$ que ens dóna $\dfrac{\partial v}{\partial x}=0$. D'aqui veïem un altre cop que $\dfrac{\partial u}{\partial x}=\dfrac{\partial u}{\partial y}=0$ que ens torna a donar que $f$ és constant.

\end{enumerate}

\item \textbf{Sigui $u(x,y)=x^{2}-y^{2}+x$. Trobeu $v(x,y)$ de manera que la funció $f(x+iy)=u(x,y)+iv(x,y)$ sigui holomorfa a $\mathbb{C}$ i compleixi $f(0)=i$.}

\underline{Resolució:}

Anem a escriure la matriu jacobiana de $f$:
$$Df(x,y)=
\left(\begin{matrix}
  u_{x}(x,y) & u_{y}(x,y) \\
  v_{x}(x,y) & v_{y}(x,y)
\end{matrix}\right) = 
\left(\begin{matrix}
  u_{x}(x,y) & u_{y}(x,y) \\
  -u_{y}(x,y) & u_{x}(x,y)
\end{matrix}\right) = $$
$$= \left(\begin{matrix}
  2x+1 & -2y \\
  2y & 2x+1
\end{matrix}\right)$$

Tenim llavors que 
\begin{enumerate}
\item $v_{x}(x,y) = 2y \Rightarrow  v(x,y) = y^{2} + c(x)$
\item $v_{y}(x,y) = 2x+1 \Rightarrow v(x,y) = x^{2} + x + c(y)$
\end{enumerate}

on $c(x)$, $c(y)$ són constants que depenen només de $x$ i de $y$ respectivament.

D'aqui podem concluïr que $v(x,y) = y^{2} + x^{2} + x + K$ on $K$ és una constant que no depèn ni de $x$ ni de $y$. Fem servir la última hipòtesi per trobar aquesta $K$:

$$f(0+0i) = i = u(0,1) + iv(0,1) \Rightarrow v(0,1) = 1$$ 

Trobem que $K=0 \Rightarrow v(x,y) = y^{2} + x^{2} + x$

\end{enumerate}

\end{document}