\documentclass[10pt,a4paper]{article}
\usepackage[utf8]{inputenc}
\usepackage[spanish]{babel}
\usepackage[a4paper]{geometry}
\usepackage{amsmath}
\usepackage{amsfonts}
\usepackage{amssymb}
\usepackage{framed}
\author{Christian José Soler}
\title{Anàlisi Complexa - Laboratori 7}
\begin{document}
\maketitle

\begin{enumerate}
\item Sigui $\Omega\subset\mathbb{C}$ una regió que conté el disc unitat tancat $\mathbb{D}$. Sigui $f$ una funció
holomorfa en $\Omega$ tal que $f(z)\neq 0$ per a $z\in\Omega\backslash\{0\}$ i a més $|f(z)| = 1$ si $|z| = 1$.
	\begin{enumerate}
	\item Proveu que existeixen $n\in\mathbb{N}\cup\{0\}$ i una funció holomorfa $g$ en $\Omega$ tals que $g(0)\neq 0$ i $f(z) = z^{n}g(z)$. 
	\begin{framed}
	
	\end{framed}
	\item Demostreu que $|g(z)| = 1$ si $z\in\mathbb{D}$.
	\begin{framed}
	
	\end{framed}
	\item Deduïu que existeix $\lambda\in\mathbb{C}$ amb $|\lambda| = 1$ tal que $f(z) = \lambda z^{n}$ si $z\in\mathbb{D}$.
	\begin{framed}
	
	\end{framed}
	\item Es compleix la igualtat en c) per a tot $z\in\Omega$?
	\begin{framed}
	
	\end{framed}
	\end{enumerate}
\item Sigui $f:\mathbb{D} \rightarrow \mathbb{D}$ una funció holomorfa tal que $f(0) = 0$. Proveu que
	\begin{enumerate}
	\item $|f(z)|\leq|z|$ per tot $z\in\mathbb{D}$ i $|f'(0)|\leq 1$
	\begin{framed}
	
	\end{framed}
	\item Si a més, $|f(z)| = |z|$ per algun $z\in\mathbb{D}\backslash\{0\}$ o bé $|f'(0)| = 1$, aleshores $f(z) = \lambda z$ per alguna $\lambda\in\mathbb{C}$ amb $|\lambda| = 1$.
	\begin{framed}
	
	\end{framed}
	\end{enumerate}			
\end{enumerate}

\end{document}